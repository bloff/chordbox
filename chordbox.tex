%% LyX 2.3.1 created this file.  For more info, see http://www.lyx.org/.
%% Do not edit unless you really know what you are doing.
\documentclass[english]{scrartcl}
\usepackage[osf]{libertineRoman}
\usepackage[osf]{biolinum}
\renewcommand{\ttdefault}{cmtt}
\usepackage[libertine]{newtxmath}
\usepackage[T1]{fontenc}
\usepackage[latin9]{inputenc}
\setcounter{tocdepth}{2}
\usepackage{color}
\usepackage{babel}
\usepackage{textcomp}
\usepackage{url}
\usepackage{microtype}
\usepackage[unicode=true,
 bookmarks=true,bookmarksnumbered=false,bookmarksopen=false,
 breaklinks=false,pdfborder={0 0 0},pdfborderstyle={},backref=false,colorlinks=false]
 {hyperref}
\hypersetup{pdftitle={Chordbox v0.1 Documentation}}

\makeatletter

%%%%%%%%%%%%%%%%%%%%%%%%%%%%%% LyX specific LaTeX commands.
\newcommand{\noun}[1]{\textsc{#1}}
%% Because html converters don't know tabularnewline
\providecommand{\tabularnewline}{\\}

%%%%%%%%%%%%%%%%%%%%%%%%%%%%%% User specified LaTeX commands.
\usepackage{chordbox}
\usepackage{filecontents}
\setkomafont{descriptionlabel}{\ttfamily\color{red}}
\newcommand\TikZ{Ti\emph{k}Z}

\begin{filecontents}{chordbox.bib}
@software{leadsheets,
 	author = {Clemens Niederberger},
  	title = {leadsheets},
  	url = {https://ctan.org/pkg/leadsheets/},
  	version = {0.5b},
  	date = {2017-09-26},
}
@software{guitarchordschemes,
 	author = {Clemens Niederberger},
  	title = {guitarchordschemes},
  	url = {https://ctan.org/pkg/guitarchordschemes/},
  	version = {0.7},
  	date = {2016-08-16},
}
@software{gchords,
 	author = {Kasper Peeters},
  	title = {gchords},
  	url = {https://ctan.org/pkg/gchords/},
  	version = {1.20},
  	date = {2008-02-03},
}
@manual{pgfmanual,
   author    = {Till Tantau},
   title     = {The TikZ and PGF Packages},
   subtitle  = {Manual for version 3.0.1a},
   url       = {http://sourceforge.net/projects/pgf/},
   date      = {2015-08-29},
}
\end{filecontents}

\makeatother

\usepackage[bibstyle=standard,citestyle=authortitle-comp]{biblatex}
\addbibresource{chordbox.bib}
\begin{document}
\titlehead{\centering\chordbox[3]{G\sharp^{maj7}}{4,x,6,5,4,3}}
\title{Chordbox}
\subtitle{A \LaTeX{} package for drawing string instrument chord diagrams}
\author{Steven Franzen}
\date{v0.1\quad{}\today}
\publishers{\url{https://github.com/sfranzen/chordbox}}

\maketitle
\tableofcontents{}

\clearpage{}

\section{Package introduction}

This package is the result of a search similar to the one undertaken
by Clemens Niederberger for his \texttt{leadsheets} \autocite{leadsheets}
package: over the years I have collected many textual guitar tabs
and chord sheets that I finally wanted to put together and typeset
properly using \LaTeX , including guitar chord diagrams. The first
part of my requirements is now more than fulfilled by \texttt{leadsheets},
which provides all the tools for putting together chords and lyrics.
For the second part I found that there were only two relevant existing
packages: \texttt{guitarchordschemes}\autocite{guitarchordschemes}
and \texttt{gchords}\autocite{gchords}. Neither of these achieved
exactly what I wanted, so this package is my own attempt at providing
that tool. It is a short collection of \TikZ{} code snippets, wrapped
in two \LaTeX{} commands that are easy to use, with several options
available to customise their output.

\subsection{License}

This work may be distributed and/or modified under the conditions
of the \LaTeX{} Project Public License, either version 1.3 of this
license or (at your option) any later version. The latest version
of this license is in \url{http://www.latex-project.org/lppl.txt}
and version 1.3 or later is part of all distributions of \LaTeX{} version
2005/12/01 or later.

\subsection{Loading the package}

\texttt{chordbox} currently does not support \LaTeX{} options, so it
is loaded simply with:

\texttt{\textbackslash usepackage\{chordbox\}}

\section{Usage}

The package provides two similar commands, \texttt{\textbackslash chordbox}
and \texttt{\textbackslash bchordbox}. The latter extends the former
to draw barred chords.

\subsection{The \texttt{\textbackslash chordbox} command}

Syntax: \texttt{\textbackslash chordbox{[}}\textlangle \emph{base
fret}\textrangle \texttt{{]}\{}\textlangle \emph{chord name}\textrangle\texttt{\}\{}\textlangle \emph{fret
positions}\textrangle\texttt{\}}

The simplest form of the command requires you to specify only the
\textlangle \emph{chord name}\textrangle{} and \textlangle \emph{fret
positions}\textrangle . The former can contain any text and math symbols
(see below), the latter should be a comma-separated list of elements,
one for each string. Each of these is interpreted as follows: a positive
number as a fretted string, the number 0 as an open string and anything
else as a muted string. It draws a grid resembling the (vertical)
strings over four horizontal frets, with a black bar at the top figuring
as the instrument's nut. The number of strings drawn is determined
by the number of \emph{fret positions} passed to the command, so generating
chord diagrams for different instruments is no problem:

\begin{tabular}{ccc}
Ukulele & 5-string banjo & 6-string guitar\tabularnewline
\texttt{\small{}\textbackslash chordbox\{Am\}\{2,0,0,0\}} & \texttt{\small{}\textbackslash chordbox\{Am\}\{x,2,2,1,2\}} & \texttt{\small{}\textbackslash chordbox\{Am\}\{x,0,2,2,1,0\}}\tabularnewline
\chordbox{Am}{2,0,0,0} & \chordbox{Am}{x,2,2,1,2} & \chordbox{Am}{x,0,2,2,1,0}\tabularnewline
\end{tabular}

By default, the \textlangle \emph{chord name}\textrangle{} is processed
by a command that typesets it in math mode, in roman font (see \texttt{\nameref{/chordbox/name}}).
This allows the use of math symbols like \texttt{\textbackslash flat}
and \texttt{\textbackslash sharp} as well as the \textasciicircum{}
(superscript) operator to typeset decorated chord names:

\begin{tabular}{cc}
\texttt{\small{}\textbackslash chordbox\{B\textbackslash flat m\}\{x,x,x,3,2,1\}} & \texttt{\small{}\textbackslash chordbox\{F\textbackslash sharp\textasciicircum\{maj7\}\}\{2,x,3,3,2,x\}}\tabularnewline
\chordbox{B\flat m}{x,x,x,3,2,1} & \chordbox{F\sharp^{maj7}}{2,x,3,3,2,x}\tabularnewline
\end{tabular}

For best results, make sure to select the same typeface for both text
and math typesetting.

The \textlangle \emph{base fret}\textrangle{} is the number that,
although formally an optional argument, must be provided for chords
extending past the number of frets in the chord box. It is used to
position the box and fretted notes relative to this fret, for example
\texttt{\textbackslash chordbox{[}6{]}\{D\textbackslash sharp m\textasciicircum\{7\textbackslash flat
5\}\}\{x,6,7,6,7,x\}}:

\chordbox[6]{D\sharp m^{7\flat 5}}{x,6,7,6,7,x}

As can be seen, the nut is not drawn in these cases.

\subsection{The \texttt{\textbackslash bchordbox} command}

Syntax: \texttt{\textbackslash bchordbox{[}}\textlangle \emph{base
fret}\textrangle \texttt{{]}\{}\textlangle \emph{chord name}\textrangle\texttt{\}\{}\textlangle \emph{fret
positions}\textrangle\texttt{\}\{}\textlangle \emph{barre frets}\textrangle\texttt{\}}

As mentioned, this command is for drawing barre chords, for which
it requires an additional comma-separated list of fret numbers. A
thick line is drawn over the string symbols at these frets. The other
arguments are identical to those of \texttt{\textbackslash chordbox},
see for example \texttt{\textbackslash bchordbox\{Bm\textasciicircum 7\}\{x,2,4,2,3,2\}\{2\}}
and \texttt{\textbackslash bchordbox{[}3{]}\{C\}\{x,3,5,5,5,3\}\{3,5\}}:

\bchordbox{Bm^7}{x,2,4,2,3,2}{2} \bchordbox[3]{C}{x,3,5,5,5,3}{3,5}

\section{Settings}

Because \texttt{chordbox} relies on \noun{pgf}/\TikZ{} for drawing,
it makes sense to use the powerful \noun{pgf} key management system
that comes with it. Therefore this package stores its code and style
settings as keys, some of which may be modified to customise the output.
This can be done anywhere in the preamble or document by means of
the the \texttt{\textbackslash pgfset} and \texttt{\textbackslash tikzset}
commands. Some examples of usage will be given below, but for more
reading about \noun{pgf} keys and their handling please refer to Section
82 of the \noun{pgf} manual\autocite{pgfmanual} where everything
is documented in full detail. In any case, modifying a key stores
its value for the \LaTeX{} group where the command is issued and its
child groups, which can override it again, but it does not propagate
up to parent groups.

\subsection{\TikZ{} keys}

The items described in this section are called \emph{styles} in \TikZ{}
terminology and contain various (lists of) keys and values that can
be applied to various drawing commands. Most importantly, the \texttt{chordbox}
style applies to the whole \texttt{\{tikzpicture\}} environment of
every chord box produced. The default setting is to scale all coordinates
by a quarter, because the drawing is done in \noun{pgf}'s ``natural''
units that, though convenient to use, result in an impractically large
picture. Note that this scaling only influences the \TikZ{} components;
in particular, fonts are not affected.

Because these keys are all stored in the \texttt{/tikz} path, it is
most convenient to change them using the \texttt{\textbackslash tikzset}
command, for example \texttt{\textbackslash tikzset\{chordbox/.style=\{scale=0.4,
thick\}\}}.
\begin{description}
\item [{/tikz/chordbox}] \hfill{}(style, initially \texttt{\{scale=0.25\}})\\
This style is applied to every \texttt{\{tikzpicture\}} environment
produced by this package.
\end{description}
The rest of the styles define the three symbol shapes used for the
fretted (\tikz[chordbox]\coordinate[string=fretted];), open (\tikz[chordbox]\coordinate[string=open];)
and muted (\tikz[chordbox]\coordinate[string=muted];) string positions.
\begin{description}
\item [{/tikz/string/base}] \hfill{}(style, initially \texttt{\{circle,
draw, inner sep=0, minimum size=20, transform shape\}})\\
Common options for all three symbols. The \texttt{transform shape}
option is required to correctly scale the symbols, which are implemented
as \TikZ{} nodes, with the rest of the picture.
\item [{/tikz/string/fretted}] \hfill{}(style, initially \texttt{\{string/base,
fill\}})
\item [{/tikz/string/open}] \hfill{}(style, initially \texttt{\{string/base,
above\}})
\item [{/tikz/string/muted}] \hfill{}(style, initially \texttt{\{string/open,
cross out, minimum size=19\}})
\end{description}

\subsection{Other keys}

The following keys are not passed directly to \TikZ{} commands, but
used to store global values. They are all in the \texttt{/chordbox}
path and should be set using the \texttt{\textbackslash pgfset} command,
for example \texttt{\textbackslash pgfset\{/chordbox/numfrets=5\}}.
\begin{description}
\item [{/chordbox/numfrets\texttt{\textcolor{black}{\small{}=\textlangle }}\texttt{\textcolor{black}{\emph{\small{}number}}}\texttt{\textcolor{black}{\small{}\textrangle{}}}}] \hfill{}(initially
\texttt{4}) \\
The number of frets drawn in each chord box.
\item [{/chordbox/name\label{/chordbox/name}}] \hfill{}(initially \texttt{\textbackslash ensuremath\{\textbackslash mathrm\{\#1\}\}})\\
The command that is used to typeset the name of the chord.
\end{description}

\appendix
\printbibliography[heading=bibintoc]

\end{document}
